\chapter{Introduction}\label{ch:introduction}
This chapter presents the motivation behind this thesis, states the main problem that is being addressed and finally
gives a brief overview of the structural design.


\section{Motivation}\label{sec:motivation}
Autonomous driving (AD) is slowly being more adopted and tested in the real world\cite{bansal2017forecasting},
but this also brings up cases
where autonomous vehicles (AV) cause crashes and other accidents.
Motion planning is a core part of AD and therefore needs thorough testing and as testing in
real world scenarios can be quite expensive and dangerous, this is mostly done in virtual simulations\cite{schoner2018simulation}.

The goal of testing is to generate many challenging scenarios and find misbehaviours in the MPA when simulating
the scenarios\cite{koopman2016challenges}.
Here most of the research focuses on testing a single MPA under simulated real world driving scenarios\cite{huang2016autonomous}.
This is very useful to prove the system can solve those scenarios and guarantee safety requirements to some
degree.
But with a rising adoption of AD, in the future there will be more AVs on the road\cite{bansal2017forecasting} and testing also
needs to focus on the interaction with other AVs, because there have already been accidents with multiple AVs involved.

This problem is addressed in this thesis, where not only test scenario generation is on focus, but also
the behaviour of multiple MPAs together in the same scenario.
Therefore, a new platform is set up to perform simulations with multiple instances of MPAs simultaneously.

The evaluation of the novel Multi-Planning-Platform (MPP) shows that both introduced
fitness functions are able to find more
critical behaviours than a baseline random approach.


\section{Problem Statement}\label{sec:problem-statement}
As AVs are becoming more prevalent, adoption is predicted to increase considerably in the future\cite{bansal2017forecasting},
and there are even trials with AV taxis\footnote{\url{https://www.roboticsbusinessreview.com/rbr50-company-2021/waymo/} Date: \FootDate}
\footnote{\url{https://www.therobotreport.com/cruise-opens-driverless-robotaxi-service-sf-public/} Date: \FootDate},
they must meet high safety requirements to avoid accidents and also to get publicly accepted.
Despite extensive testing, both in the real world and in simulations, there are still accidents in
which AVs are involved and often even at fault
\footnote{\url{https://www.businessinsider.com/self-driving-car-accidents-waymo-cruise-tesla-zoox-san-francisco-2022-1} Date: \FootDate}.
A big factor here is the test cases in which AVs are repeatedly tested with simulated traffic participants.
However, this neglects how AVs interact with each other;
these interactions are completely unpredictable and can lead to a variety of situations.
Furthermore, the more AVs are adapted in the future, the more relevant this issue will become.
There are already actual cases of AVs interacting with each other in an unforeseeable way,
causing accidents\footnote{\url{https://edition.cnn.com/2022/09/01/business/cruise-robotaxi-recall/index.html}\\Date: \FootDate}
or roadblocks\footnote{\url{https://www.theverge.com/2022/7/1/23191045/cruise-robotaxis-driverless-roadblock-san-francisco} Date: \FootDate}.
In order to avoid such interactions and to investigate them in advance,
the behaviour of AVs among each other should specifically be tested.
So this thesis is focused on testing interactions between motion planners
and the main objective is to find critical interactions, specifically collisions,
disengagements and missed goals.
In this process the novel MPP is developed for simulation purposes
and coupled with an evolutionary algorithm to generate actual critical scenarios.
Critical metrics are tracked during simulations in order to answer which of the introduced
evolutionary approaches is the best
to find critical interactions in motion planners.

\section{Overview}\label{sec:overview}
First of all, the~\nameref{ch:background} chapter gives an overview of the current
state of the art for testing autonomous vehicles and MPAs,
then the technologies and concepts used for implementation are explained, this includes an explanation of the
\CommonRoad~framework and the idea of coupling it with an evolutionary algorithm.
In the next chapter,~\nameref{ch:methodology}, the idea behind the MPP and
several details about the design of the individual components follow.
Then, in the~\nameref{ch:evaluation} chapter, the experimental setup and the
research questions are first defined and then addressed in the analysis,
where each question is examined quantitatively and qualitatively.
Finally, the~\nameref{ch:conclusion-future-work} chapter provides a summary of the findings of this thesis and
what learnings can be drawn from it, and lastly gives an outlook on future challenges.
