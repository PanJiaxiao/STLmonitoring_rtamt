\chapter{Evaluation}\label{ch:evaluation}

% variables
\newcommand{\corInteractionsCollisions}{0.15}
\newcommand{\corPInteractionsCollisions}{\ll~0.05}
\newcommand{\corInteractionsEmergency}{0.12}
\newcommand{\corPInteractionsEmergency}{\ll~0.05}
\newcommand{\corInteractionsGoals}{0.22}
\newcommand{\corPInteractionsGoals}{\ll~0.05}
\newcommand{\corEmergencyGoals}{0.82}

\newcommand{\corDangerInteractionsCollisions}{0.20}
\newcommand{\corInteractionsInteractionsCollisions}{0.19}
\newcommand{\corRandomInteractionsCollisions}{0.19}

\newcommand{\corDangerInteractionsEmergency}{0.17}
\newcommand{\corInteractionsInteractionsEmergency}{0.21}
\newcommand{\corRandomInteractionsEmergency}{0.15}

\newcommand{\corDangerInteractionsGoals}{0.18}
\newcommand{\corInteractionsInteractionsGoals}{0.40}
\newcommand{\corRandomInteractionsGoals}{0.07}

\newcommand{\aOneTwoDangerInteractionsCollisions}{0.60}
\newcommand{\aOneTwoDangerInteractionsEmergency}{0.88}
\newcommand{\aOneTwoDangerInteractionsGoals}{0.86}
\newcommand{\aOneTwoDangerInteractionsInteractions}{0.11}
\newcommand{\aOneTwoDangerRandomCollisions}{0.83}
\newcommand{\aOneTwoDangerRandomEmergency}{0.99}
\newcommand{\aOneTwoDangerRandomGoals}{0.95}
\newcommand{\aOneTwoDangerRandomInteractions}{1}
\newcommand{\aOneTwoInteractionsDangerCollisions}{0.40}
\newcommand{\aOneTwoInteractionsDangerEmergency}{0.12}
\newcommand{\aOneTwoInteractionsDangerGoals}{0.15}
\newcommand{\aOneTwoInteractionsDangerInteractions}{0.89}
\newcommand{\aOneTwoInteractionsRandomCollisions}{0.87}
\newcommand{\aOneTwoInteractionsRandomEmergency}{0.97}
\newcommand{\aOneTwoInteractionsRandomGoals}{0.90}
\newcommand{\aOneTwoInteractionsRandomInteractions}{1}

\newcommand{\pDangerInteractionsCollisions}{0.31}
\newcommand{\pDangerInteractionsEmergency}{\ll~0.05}
\newcommand{\pDangerInteractionsGoals}{\ll~0.05}
\newcommand{\pDangerInteractionsInteractions}{\ll~0.05}
\newcommand{\pDangerRandomCollisions}{\ll~0.05}
\newcommand{\pDangerRandomEmergency}{\ll~0.05}
\newcommand{\pDangerRandomGoals}{\ll~0.05}
\newcommand{\pDangerRandomInteractions}{\ll~0.05}
\newcommand{\pInteractionsRandomCollisions}{\ll~0.05}
\newcommand{\pInteractionsRandomEmergency}{\ll~0.05}
\newcommand{\pInteractionsRandomGoals}{\ll~0.05}
\newcommand{\pInteractionsRandomInteractions}{\ll~0.05}

\newcommand{\numberRuns}{60}
\newcommand{\runsPerScenarioPerAlgorithm}{5}
\newcommand{\runsPerScenario}{15}
\newcommand{\runsPerAlgorithm}{20}
\newcommand{\numberScenarios}{4}
\newcommand{\numberSimulationsPerRun}{400}


\section{Experimental Setup}\label{sec:experimental-setup}
To evaluate the novel multi-motion-planning-system the three algorithms were tested on four different
scenarios taken from the \CommonRoad~scenario library\footnote{\url{https://commonroad.in.tum.de/scenarios} Date: \FootDate}.
The experiment consists of several runs, whereby the base scenarios were simulated \numberSimulationsPerRun~times per run and
mutations are applied between the individual simulations according to the chosen algorithm.
For each scenario and algorithm combination \runsPerScenarioPerAlgorithm~runs were carried out,
this lead to overall \runsPerScenario~runs per scenario and \runsPerAlgorithm~runs per algorithm, overall \numberRuns~runs
and therefore 24.000 individual simulations.
With the configuration~\ref{tab:table:experiment-config} a single run took about 3--8 hours, depending on the used algorithm
and the scenario.
The experiments were carried out on a Ryzen5 3600 3.8GHz CPU with 16GB RAM\@.
Runs could be triggered in parallel for different scenarios and algorithms to improve the overall experiment
duration.

To find a good configuration for the final study, a pilot study was conducted in advance.
% TODO also include a diagram here to show that values are not growing after set amount of simulations?
% TODO explain each configuration value?

\begin{table}[h!]
    \setlength{\tabcolsep}{3pt} % default value: 6pt
    \begin{tabularx}{\textwidth}{msb}
        \toprule
        \textbf{Configuration} & \textbf{Value} & \textbf{Explanation}
        \\
        \midrule
        \texttt{simulation} \texttt{\_retries} & 400 & number of simulations per run\\
        \addlinespace
        \addlinespace
        \texttt{dist\_to\_end} & 10.0 & length of goal area at the end of a lanelet\\
        \addlinespace
        \addlinespace
        \texttt{max\_amount\_pp} & 6 & maximum amount of planning problems in a scenario\\
        \addlinespace
        \addlinespace
        \texttt{min\_amount\_pp} & 2 & minimum amount of planning problems in a scenario\\
        \addlinespace
        \addlinespace
        \texttt{max\_position} \texttt{\_modifier} & 50 & maximum amount for position mutations, minimum amount is the negative value of this configuration\\
        \addlinespace
        \addlinespace
        \texttt{max\_velocity} & 25 & maximum velocity of an ego after mutation\\
        \addlinespace
        \addlinespace
        \texttt{min\_velocity} & 5 & minimum velocity of an ego after mutation\\
        \addlinespace
        \addlinespace
        \texttt{max\_init\_velocity} & 10 & maximum initial velocity for new planning problems and also maximum modifier for velocity mutations\\
        \addlinespace
        \addlinespace
        \texttt{min\_init\_velocity} & 5 & minimum initial velocity and minimum modifier for velocity mutations\\
        \addlinespace
        \addlinespace
        \texttt{max\_time\_step} & 200 & maximum number of time step that each planner can take in a simulation\\
        \addlinespace
        \addlinespace
        \texttt{emergency\_slow} \texttt{\_down\_amount} & 5 & how much a planner is slowing down for each time step it spends in emergency mode\\
        \bottomrule
    \end{tabularx}
    \caption{Configuration values of \texttt{config.py} for the experiment.}
    \label{tab:table:experiment-config}
\end{table}

\subsection{Base Scenario Selection}\label{subsec:base-scenario-selection}
For the evaluation four highway specific scenarios were chosen, as the MPA (Reactive Planner~\ref{subsubsec:reactive-planner})
was designed towards this kind of scenarios.
The scenarios were selected by features from \CommonRoad's online library, mentioned in \autoref{sec:experimental-setup}.
The scenarios have different features to cover the basic highway driving maneuvers and to test how
the different road networks affect the planners' behaviour.

\textbf{Ramp Scenario (RS}~\autoref{fig:scenario:rs}\textbf{)}: The ramp scenario has one on-ramp and two off-ramps onto
a regular two-lane highway.
The on-/off-ramps are designed to test whether planners can correctly arrange themselves in the lanes
in order to reach the goal on an off-ramp or on the highway when they are finished.

\textbf{Merge Scenario (MS}~\autoref{fig:scenario:ms}\textbf{)}: The merge scenario has long almost straight roads
that merge together.
The scenario aims to generate critical situations during the merge and tailgating interactions on the long
sections before and after.

\textbf{Multi Lane Scenario (MLS}~\autoref{fig:scenario:mls}\textbf{)}: This scenario has several lanes next to each other
and should generate a lot of interaction between the planners as they try to place themselves on their
respective goal lane and thereby also influence other planners.

\textbf{Oncoming Lane Scenario (OLS}~\autoref{fig:scenario:ols}\textbf{)}: This scenario has oncoming lanes, which
should ultimately lead to serious critical situations such as collisions.


\begin{figure}
    \centering
    \includegraphics[width=0.82\linewidth]{images/scenarios/USA_US101-23_1_T-1}
    \caption{Highway scenario with on and off ramps, Ramp Scenario [RS]}
    \label{fig:scenario:rs}
\end{figure}
\begin{figure}
    \centering
    \includegraphics[width=0.82\linewidth]{images/scenarios/DEU_Cologne-79_2_I-1}
    \caption{Scenario with oncoming traffic lanes, lane splits and merges, Oncoming Lane Scenario [OLS]}
    \label{fig:scenario:ols}
\end{figure}
\begin{figure}
    \centering
    \includegraphics[width=0.82\linewidth]{images/scenarios/CHN_Cho-2_2_l-1-1}
    \caption{Scenario with long straight roads and a lane merge, Merge Scenario [MS]}
    \label{fig:scenario:ms}
\end{figure}
\begin{figure}
    \centering
    \includegraphics[width=0.82\linewidth]{images/scenarios/USA_US101-15_2_T-1}
    \caption{Scenario with multiple lanes, Multi Lane Scenario [MLS]}
    \label{fig:scenario:mls}
\end{figure}

\newpage
\section{Results}\label{sec:results}
\subsection{Research Questions}\label{subsec:research-questions}

\textbf{RQ 1 (Failure Exposure):} \textit{Can the system automatically detect faults that are due to interactions?}

% TODO not sure if needed
%\textbf{RQ 1.1} \textit{Is there a difference for the algorithms?}
%
%\textbf{RQ 1.2} \textit{Is there a difference for the scenarios?}

\textbf{RQ 2 (Algorithm Comparison):} \textit{Which fitness metric is suited best for generating critical interactions in scenarios?}

\textbf{RQ 3 (Effect of Scenario Features):} \textit{Are there features within the scenarios that promote different metrics in the simulations?}



\subsection{Analysis}\label{subsec:analysis}

\subsubsection{Side Notes: Qualitative Analysis}
In this case the qualitative analysis deals with the GIFs of the individual simulations.
For this purpose, simulations are selected and analysed on the basis of conspicuous critical
metrics within the data of a run.
This is necessary because on the basis of the critical metrics alone
it is not possible to draw conclusions about the exact course and behaviour of the planners during a simulation.

In the qualitative analysis at the end of the experiment, two main categories of interactions were found,
firstly Tailgating Interactions and secondly Lane Switch Interactions.
These occur in most scenarios and produce similar behavioural patterns that result in the critical metrics.
These two interactions are described below and will be referred to in the qualitative analyses of the individual
detailed analyses.

\paragraph{Tailgating Interaction}\label{par:tailgating-interaction}\mbox{}\\
% TODO add images for different tailgating interactions or actually the one that gives lots of interactions
A tailgating interaction is considered as two or more ego cars going behind each other
with the ego(s) in the back going slightly faster than the one in front.
This behaviour causes the egos to drive very close to each other and due to the higher velocity from the ego
in the back also generates interactions.
These tailgating interactions can have these different outcomes:
\begin{enumerate}
    \item The ego in front has enough space to move aside and let the ego in the back pass, this only leads to a
            few interactions.
    \item The ego in front can not find a solution due to the ego in the back being too close
            and switches to emergency mode which slows down the ego, this
            often leads to a collision due to the close distances and the immediate slowing
            down of the ego in front.
    \item The ego in front is continuously adjusting its trajectory which counts as lots of interactions but there
            is not enough space for the ego in the back to overtake, so this boosts the scenarios' interaction count
            by a lot without actually having a dangerous interaction.
            % TODO is this manually reproducible relevant?
            This interaction is not interesting for the experiment and is also easily manually reproducible by
            putting two planning problems in the same lane with the same goal.
\end{enumerate}

\paragraph{Lane Switch Interaction}\label{par:laneswitch-interaction}\mbox{}\\
A lane switching interaction describes two or more egos driving in lanes approximately next to each other
and at least one ego trying to switch lanes in the process.
This means that at least one of the planners involved must adjust their trajectory to avoid a collision
and thereby leads to interactions.
These lane change interactions can lead to various interactions:
\begin{enumerate}
    \item The egos involved are slightly offset with about the same velocities
        and can therefore be placed behind each other with slight
        adjustments to the trajectory, only a few interactions occur in this process.
    \item One ego has a higher velocity and is cut off by another, this leads
        to several interactions and potentially \EM~or \Collisions~as well.
    \item Egos travel at the same height at similar velocities and therefore no lane
        change can take place, this leads to many interactions and often \Goals~or \EM.
\end{enumerate}

\newpage


\subsubsection{RQ1: Failure Exposure}\label{subsubsec:failure-exposure}
This research question explores the relationship between the defined interaction metric and the critical metrics
(\EM, \Collisions, \Goals).
This aims to show to what extent critical situations are caused by interactions.

\begin{figure}
    \centering
    \begin{tabular}{ |c|c|c| }
        \hline
        metric & correlation & p-value\\
        \hline
        \EM & \corInteractionsEmergency & < 0.05\\
        \Goals & \corInteractionsGoals & < 0.05\\
        \Collisions & \corInteractionsCollisions & < 0.05\\
        \hline
    \end{tabular}
    \caption{Correlation values of fault metrics compared to interactions metric.}
    \label{fig:correlation-interactions}
\end{figure}

\begin{figure}
    \centering
    \includegraphics[width=0.7\linewidth]{images/plots/scatter-emergency-interactions}
    \caption{Relationship between \EM~and \Interactions~across all individual simulations.}
    \label{fig:plots:scatter-emergency}
\end{figure}

\begin{figure}
    \centering
    \includegraphics[width=0.7\linewidth]{images/plots/scatter-goals-interactions}
    \caption{Relationship between \Goals~and \Interactions~across all individual simulations.}
    \label{fig:plots:scatter-goals}
\end{figure}

\begin{figure}
    \centering
    \includegraphics[width=0.7\linewidth]{images/plots/scatter-collisions-interactions}
    \caption{Relationship between \Collisions~and \Interactions~across all individual simulations.}
    \label{fig:plot:scatter-collisions}
\end{figure}

%TODO also analyze correlations with danger metric?

For this research question all the collected data needs to be analysed together in order to test whether
there is a correlation between the interactions and the critical metrics.
The analysis shows that there is a weak correlation for all critical metrics and all of them have a significant
p-value (\autoref{fig:correlation-interactions}).

The metric \EM~only shows a very weak correlation with interactions (cor: \corInteractionsEmergency, p-value).
This can be explained by the behaviour of the planner, which in many situations can not find a solution
for the state of the ego car and therefore going into emergency mode.
Meaning there are egos in emergency mode without having interactions with other ego cars.

The metric \Goals~has the highest correlation with interactions.
The main reason ego cars do not reach the goal is that the ego stopped during the simulation because of too
many steps spent in emergency mode and then stopping on the road.
Another reason can be that there were not enough steps in the simulation to drive from start to goal, but
this can mostly be neglected due to the pilot study where cars mostly reached the finish. %TODO sketchy sentence
This is also backed by the correlation between \EM~and \Goals~(cor: 0.79, p < 0.05).
%TODO how can this be pulled back on interactions

The metric collisions shows a weak correlation with interactions (cor: \corInteractionsCollisions, p-value < 0.05).
This means that interactions rarely lead to collisions.

\noindent\fbox{\parbox{\textwidth}{\textbf{Summary:} The system can automatically detect faults that are due to interactions, but
interactions are not a very good metric on their own to generate critical situations, as an interaction can
also take place with sufficient safety distance.}}

\newpage

\subsubsection{RQ2: Algorithm Comparison}\label{subsubsec:algorithm-comparison}
This research question investigates which of the fitness metrics, danger or interactions,
can generate the most critical metrics in the simulations.
First, a comparison is made with the basic \Random~approach and then the two algorithms
\MaxDanger~and \MaxInteractions~are compared with each other.


\begin{figure}
    \centering
    \includegraphics[width=0.7\linewidth]{images/plots/emergency_maneuvers_per_algorithm}
    \caption{Aggregated \EM~per run for algorithms \MaxDanger, \MaxInteractions~and \Random.}
    \label{fig:plots:emergency_maneuvers_per_algorithm}
\end{figure}

\begin{figure}
    \centering
    \includegraphics[width=0.7\linewidth]{images/plots/goals_not_reached_per_algorithm}
    \caption{Aggregated \Goals~per run for algorithms \MaxDanger, \MaxInteractions~and \Random.}
    \label{fig:plots:goals_not_reached_per_algorithm}
\end{figure}

\begin{figure}
    \centering
    \includegraphics[width=0.7\linewidth]{images/plots/collisions_per_algorithm}
    \caption{Aggregated \Collisions~per run for algorithms \MaxDanger, \MaxInteractions~and \Random.}
    \label{fig:plot:collisions_per_algorithm}
\end{figure}

\begin{figure}
    \centering
    \includegraphics[width=0.7\linewidth]{images/plots/interactions_per_algorithm}
    \caption{Aggregated \Interactions~per run for algorithms \MaxDanger, \MaxInteractions~and \Random.}
    \label{fig:plots:interactions_per_algorithm}
\end{figure}


\begin{figure}
    \centering
    \begin{tabular}{ |c|c| c| c| }
        \hline
        algorithm & metric & p-value & A12 effect size\\
        \hline
        max\_danger & emergency\_maneuvers & \pDangerRandomEmergency &  \aOneTwoDangerRandomEmergency\\
        & goals\_not\_reached & \pDangerRandomGoals &\aOneTwoDangerRandomGoals\\
        & collisions & \pDangerRandomCollisions&\aOneTwoDangerRandomCollisions\\
        & interactions & \pDangerRandomInteractions &\aOneTwoDangerRandomInteractions\\
        max\_interactions & emergency\_maneuvers & \pInteractionsRandomEmergency & \aOneTwoInteractionsRandomEmergency\\
        & goals\_not\_reached & \pInteractionsRandomGoals &\aOneTwoInteractionsRandomGoals\\
        & collisions & \pInteractionsRandomCollisions &\aOneTwoInteractionsRandomCollisions\\
        & interactions & \pInteractionsRandomInteractions &\aOneTwoInteractionsRandomInteractions\\
        \hline
    \end{tabular}
    \caption{Comparison of algorithms \texttt{max\_danger} and \texttt{max\_interactions} to baseline approach \texttt{random}.}
    \label{fig:statistical-algorithm-comparison}
\end{figure}

\subsubsection{General Observations}
At first sight, there is a notable similarity between the graphs for
\EM\linebreak(\cref{fig:plots:emergency_maneuvers_per_algorithm})
and \Goals~(\cref{fig:plots:goals_not_reached_per_algorithm}), this is because the metric \linebreak
\Goals~is directly dependent on \EM, since after all emergency maneuvers of a planner are used up,
the ego vehicle simply stops on the road and thus cannot reach the goal.
The metric \Collisions~affects \Goals~in the same way, as two planners that have collided can no longer reach the goal area,
however, the effects on the graphs are not visible because collisions are not very frequent.

\subsubsection{Algorithm \Random}
This mutation approach purely serves as a baseline comparison for the algorithms \texttt{max\_danger} and \texttt{max\_interactions}.
The main purpose was to properly evaluate the other implementations and help to prove that those are good
drivers to generate critical interactions.

\paragraph{Quantitative Analysis}\mbox{}\\
Figures \cref{fig:plots:emergency_maneuvers_per_algorithm,fig:plots:goals_not_reached_per_algorithm,fig:plot:collisions_per_algorithm}
show that the algorithm \texttt{random} is clearly the worst performer.
This is also confirmed by the statistical values in \cref{fig:statistical-algorithm-comparison}, as the A12 effect
sizes of the algorithms \MaxDanger~and \MaxInteractions~are significantly better for
all critical metrics and also for the interaction metric.
%todo why is this so? because random just can not find anything good with random distribution all over the lanelet
 % network


\paragraph{Qualitative Analysis}\mbox{}\\
As bad as the overall results of the algorithm \Random~are,
over a whole run with enough time steps, as in the configuration, the algorithm is often able
to find a few scenarios with good critical values.
%TODO explain WHYYYY this is that way

Manual analysis of the simulated GIFs shows that for the case of high interactions those are coming from randomly
found tailgating interactions (3) of two (sometimes more) egos over a long distance of a single lanelet in the scenario.

\noindent\fbox{\parbox{\textwidth}{\textbf{Summary:} The qualitative analysis has shown that algorithm \Random
only rarely finds good critical scenarios. When it finds a scenario with high critical metrics it often is an
uninteresting~\nameref{par:tailgating-interaction} (3).}}

\subsubsection{Algorithm \MaxDanger}\label{subsubsec:algo-max_danger}

\paragraph{Quantitative Analysis}\mbox{}\\
The \MaxDanger~algorithm is significantly better than random in all metrics (\cref{fig:plots:emergency_maneuvers_per_algorithm,fig:plots:goals_not_reached_per_algorithm,fig:plot:collisions_per_algorithm}).
Overall, \MaxDanger~performs best in all critical metrics and is only beaten in the interaction metric by the
algorithm \MaxInteractions~\cref{fig:plots:interactions_per_algorithm}.
For the metric \EM~(p-value \pDangerInteractionsEmergency, A12: \aOneTwoDangerInteractionsEmergency) and
\Goals~(p-value \pDangerInteractionsGoals, A12: \aOneTwoDangerInteractionsGoals) algorithm \MaxDanger~performs
significantly better than \MaxInteractions, for \Collisions~the two algorithms are almost even
(p-value: \pDangerInteractionsCollisions, A12: \aOneTwoDangerInteractionsCollisions) and for the metric \Interactions~
it is significantly worse than algorithm \MaxInteractions
(p-value \pDangerInteractionsInteractions, A12: \aOneTwoDangerInteractionsInteractions).
These statistics make the approach \MaxDanger~the overall best algorithm to generate critical driving metrics.
The baseline mutation \texttt{random} is outperformed significantly
in all relevant measures, emergency maneuvers, goals not reached,
collisions and interactions.

The differences in the critical metrics can be traced back to the definition of the danger fitness metric,
which favours small distances between planners and high relative velocities (\cref{subsubsec:danger-def}).
From a purely physical point of view, this makes it difficult for the planners to find a solution and
in the end this leads to most critical metrics.



% more pages: Other interesting statistics (like are there more egos, higher velocities, more steps taken)

\begin{figure}
    \centering
    \includegraphics[width=1\linewidth]{images/showcase-gifs/max_danger_us23_tailgating_fast/frame_000_delay-0.01s}
    \caption{From the left side three planners approach three other planners which are slower.}
    \label{fig:showcase-max-danger:tailgating-us23:initial}
\end{figure}

\begin{figure}
    \centering
    \includegraphics[width=1\linewidth]{images/showcase-gifs/max_danger_us23_tailgating_fast/frame_078_delay-0.01s}
    \caption{Critical point at which the purple planner gives way to the blue one.}
    \label{fig:showcase-max-danger:tailgating-us23:intermediate}
\end{figure}

\begin{figure}
    \centering
    \label{fig:showcase-max-danger:tailgating-us23:finished}
    \includegraphics[width=1\linewidth]{images/showcase-gifs/max_danger_us23_tailgating_fast/frame_139_delay-0.01s}
    \caption{Final state of the simulation with the collision of the blue and green planner and the non-achieved goal
    of the purple planner.}
\end{figure}

\paragraph{Qualitative Analysis}\mbox{}\\
Manual analysis of the generated GIFs for each simulation during a run shows that due to the
danger metric(\cref{subsubsec:danger-def})
the ego cars tend to drive very close to each other.
This supports the statistics from the quantitative analysis.

The example from \cref{fig:showcase-max-danger:tailgating-us23:finished} shows a tailgating interaction inside a
simulation.
The initial scenario~\cref{fig:showcase-max-danger:tailgating-us23:initial}
shows fast planners from the left side distributed over all three lanes and in front of them
planners with slower speed also distributed over all three lanes.
Based on this setup, one can already see that at a certain point the faster planners will catch up with the slower
ones and interactions will take place.
When the rear planners catch up with the front planners, they change lanes to let the faster planners pass.
Shortly before the finish and the off ramp, the blue planner reaches the purple one, which then changes lanes
(\nameref{par:tailgating-interaction}(1)).
However, when the blue planner reaches the green planner, the latter has no possibility to
move out of the way and there is a collision between the blue and green planner (\nameref{par:tailgating-interaction}(2)).
In the end, the purple planner cannot reach the goal area either, because the road is blocked by the collision.

Another observation of the simulated scenarios is that planners with the algorithm \MaxDanger~sometimes do not start
and only use up the emergency maneuvers until they then stop.
This has been shown to be an issue with the planner, but in simulations with the \MaxDanger~algorithm, it often leads
to planners blocking a road from the beginning of a simulation, as other planners have to drive around the static
obstacle and this is gets a good score from the danger metric (high relative velocity, close distances).
However, these scenarios are not very interesting and do not generate much interactions.
Under these circumstances, it would also be possible to directly insert static obstacles, which is not the scope
of this work.

\noindent\fbox{\parbox{\textwidth}{\textbf{Summary:} The quantitative analysis has shown that the \MaxDanger~algorithm produces the highest number of critical metrics.
But the qualitative analysis shows that the scenarios with high danger values are not necessarily the best for critical interactions and also bring problems.}}
%TODO result of max-danger is also good in the qualitative analysis, change result sentence here


\subsubsection{Algorithm \MaxInteractions}
The idea of the \MaxInteractions~algorithm is to maximise the interactions and thereby obtain simulations with many
critical situations.

\paragraph{Quantitative Analysis}\mbox{}\\
The \MaxInteractions~algorithm performs significantly
better than \Random~in all critical metrics, and also in the interaction metric
(\cref{fig:statistical-algorithm-comparison},
\cref{fig:plots:emergency_maneuvers_per_algorithm,fig:plots:goals_not_reached_per_algorithm,fig:plot:collisions_per_algorithm}).
This means that the introduced definition of an interaction (\cref{subsubsec:interaction-def})
promotes critical metrics and finds more of them than a random approach.

As already mentioned in the analysis of algorithm \MaxDanger, the algorithm\newline \MaxInteractions~performs
worse in the critical metrics, but is significantly better in the interaction metric (\cref{subsubsec:algo-max_danger}),
which is simply due to the fact that this metric is preferred by the fitness function.
This is because the danger function promotes small distances and great differences in velocities.
According to the definition of an interaction, it is sufficient here if a planner changes its trajectory and another
planner is nearby.
For this to happen, the planners do not necessarily have to be very close to each other and there does not have
to be a difference in velocity, which, in the case of \MaxDanger, ultimately brings them even closer
together over several time steps.
One reason for the worse performance on the critical metrics is that the \Collisions~and \EM~metrics result in a
standstill of the planners and therefore the planners can not generate any further interactions, which is badly rated
by the \MaxInteractions~algorithm.
%TODO move sentence below and maybe adjust sentence above
While this does not bring the planners as close and does not generate as many critical metrics, it does encourage
a different type of interactions, namely lane switch interaction (\cref{par:laneswitch-interaction}).


\paragraph{Qualitative Analysis}\mbox{}\\
The analysis of the GIFs shows that the planners often get very close by switching lanes and thus generate interactions,
but these are not as critical as those from the \MaxDanger~algorithm.
This is because the speed differences are not that high and planners have to constantly adjust their trajectory
to avoid colliding sideways, which is badly rated by the algorithm.

There are also planners with \MaxInteractions~who do not find a solution from the beginning, but this does not occur
so often because these planners cannot generate any interactions themselves and only generate a few interactions for
other planners.

The example from \cref{fig:showcase-max-interactions:laneswitch-us23:initial}~shows several lane switching interactions.
In the initial scenario \cref{fig:showcase-max-interactions:laneswitch-us23:initial},
three planners are distributed next to each other on the lanes and most of them have the exit at the end of the scenario
as goal area.
In the course of the simulation,
more and more planners try to get into the lane of the exit, which leads to the lane switching interactions.
In \cref{fig:showcase-max-interactions:laneswitch-us23:intermediate}, the purple planner switches lanes
into the very narrow gap between the blue and green planner, which leads to trajectory adjustments for
all three planners involved.
As can also be seen in \cref{fig:showcase-max-interactions:laneswitch-us23:intermediate},
the green planner approaches and tries to get into the exit lane as well.
However, since the three other planners are already very close together on this lane,
the green planner finds no solution for a lane change and then comes to a standstill in emergency mode
\cref{fig:showcase-max-interactions:laneswitch-us23:finished}.
%TODO refs to laneswitch explanation

\noindent\fbox{\parbox{\textwidth}{\textbf{Summary:} The quantitative analysis has shown that the \MaxInteractions~algorithm generates
more critical metrics than the \Random~algorithm, but less than \MaxDanger.
For producing many \Interactions~the algorithm \MaxInteractions~is the best approach.
The qualitative analysis has shown that \MaxInteractions~generates a high quality of scenarios that yield
interesting interactions, mostly due to lane switching.}}


\begin{figure}
    \centering
    \includegraphics[width=1\linewidth]{images/showcase-gifs/max_interactions_us23_lane_switch/frame_000_delay-0.01s}
    \caption{There are two pairs of planners on lanes that are directly adjacent to each other and have the same goal
    area, namely the exit ramp at the end of the lanelet network.}
    \label{fig:showcase-max-interactions:laneswitch-us23:initial}
\end{figure}
\begin{figure}
    \centering
    \label{fig:showcase-max-interactions:laneswitch-us23:intermediate}
    \includegraphics[width=1\linewidth]{images/showcase-gifs/max_interactions_us23_lane_switch/frame_090_delay-0.01s}
    \caption{Critical point at which the purple planner joins the blue and green planners in order to reach the exit.
    The green planner also tries to reach this exit.}
\end{figure}
\begin{figure}
    \centering
    \label{fig:showcase-max-interactions:laneswitch-us23:finished}
    \includegraphics[width=1\linewidth]{images/showcase-gifs/max_interactions_us23_lane_switch/frame_129_delay-0.01s}
    \caption{The green planner cannot find a solution and cannot reach the goal area because the three planners
    on the exit lane are in the way. All the other planners reach their goal area.}
\end{figure}

\newpage

\subsubsection{RQ3: Effect of Scenario Features}\label{subsubsec:scenario-effects}
This research question investigates whether road features from the baseline scenarios have an impact on the
measured critical metrics.
Furthermore, this will test the planner and MPP under a wider range of driving situations and provide insights
into test generation and simulation under different conditions.


\begin{figure}
    \centering
    \includegraphics[width=0.7\linewidth]{images/plots/scenarios_interactions}
    \caption{Comparison of the metric \Interactions~across all scenarios split by fitness function}
    \label{fig:plots:scenarios_interactions}
\end{figure}
\begin{figure}
    \centering
    \includegraphics[width=0.7\linewidth]{images/plots/scenarios_emergency}
    \caption{Comparison of the metric \EM~across all scenarios split by fitness function}
    \label{fig:plots:scenarios_emergency}
\end{figure}
\begin{figure}
    \centering
    \includegraphics[width=0.7\linewidth]{images/plots/scenarios_goals}
    \caption{Comparison of the metric \Goals~across all scenarios split by fitness function}
    \label{fig:plots:scenarios_goals}
\end{figure}
\begin{figure}
    \centering
    \includegraphics[width=0.7\linewidth]{images/plots/scenarios_collisions}
    \caption{Comparison of the metric \Collisions~across all scenarios split by fitness function}
    \label{fig:plots:scenarios_collisions}
\end{figure}


\paragraph{Quantitative Analysis}\mbox{}\\
Since hardly any statistically relevant differences were found, the quantitative analysis is presented separately
in advance.

The quantitative analysis in this case refers to all critical metrics as well as the interaction metric,
which are grouped by scenario;
the algorithms do not factor into the statistical analysis.

For all metrics examined, only the \Goals~metric showed a significant difference for the OLS;
all other metrics did not show large or significant differences for any of the scenarios.
The OLS scores significantly less \Goals~than MS (p-value: 0.031, A12: 0.73) and RS (p-value: 0.025, A12: 0.74),
MLS (p-value: 0.089, A12: 0.68) is not quite significantly better.

Even though the difference is not significant, the MLS and RS scenarios, as suspected in advance,
generate the most interactions due to the possibility of many lane changes with the multiple
lanes and on-/off-ramps next to each other~\ref{fig:plots:scenarios_interactions}.

The values for the metrics collisions and \EM~are very balanced.
An interesting aspect here is that the values if, in addition to the grouping by scenarios,
one also divides them according to the algorithms.
The MS scenario achieves many collisions at \MaxDanger and almost none at \MaxInteractions,
whereas the opposite is true for OLS~\ref{fig:plots:scenarios_collisions}.
This shows that the simulations can behave completely differently depending on the algorithm and the scenario.


\noindent\fbox{\parbox{\textwidth}{\textbf{Summary:} The quantitative analysis has shown that road features of the scenarios
have an influence on the metrics, but statistically this is only relevant to a limited extent. The combination
of scenarios and fitness functions can have interesting impacts on the results.}}


\subsubsection{Analysis for Multi Lane Scenario [MLS]}
The MLS scenario aimed to generate many interactions with the lanes next to each other, as lane
changes are possible at almost any point in this scenario.
The lane changes should then also generate critical maneuvers.

\paragraph{Qualitative Analysis}\mbox{}\\
The many interactions are caused by planners trying to change lanes on adjacent roads and thus taking
each other's space away.

Most critical situations arise due to several planners driving side by side and crossing paths on the way to their goals.
With two planners, a solution can often be found, but when even more planners are involved,
the situation can become very chaotic and planners drive extremely close to each other and hardly stay in one lane.

\noindent\fbox{\parbox{\textwidth}{\textbf{Summary:} The qualitative analysis confirms the
expected high number of interactions and shows chaotic situations in which planners often do not find a solution
due to conflicting routes.}}

\subsubsection{Analysis for Ramp Scenario [RS]}
The scenario should generate many interactions similar to the MLS and have more critical interactions with
the on-ramps/off-ramps as planners need to get into the turn lane in time.

\paragraph{Qualitative Analysis}\mbox{}\\
The goal with the critical situations at the on and off ramps has been achieved.
Often planners do not manage to get into the on/off lane in time and, after missing the off-ramp, can no longer
find a solution.
The planners then come to a standstill next to the exit and block the road for other planners.
Often there are \Collisions~or many \EM~when planners try to change to the exit lane at the last moment.

Furthermore, many critical situations arise during normal lane changes similar to the MLS,
so that the on/off ramps are not always relevant.


\subsubsection{Analysis for Merge Scenario [MS]}
The scenario was chosen to create critical situations during the merge and to generate tailgating interactions
on the long straights.

\paragraph{Qualitative Analysis}\mbox{}\\
There are often tailgating interactions in which the planners come to a standstill due to the \EM, as
described in~\nameref{par:tailgating-interaction} (3).
As a result, the roads are blocked and since there are hardly any possibilities to take avoiding actions,
the following planners are severely hindered, which leads to more critical situations.

The merge rarely has an influence on the course of the scenario, as often one of the two roads
is blocked by a planner anyway.


\noindent\fbox{\parbox{\textwidth}{\textbf{Summary:} The MS does not produce many interesting situations compared to
the other scenarios and the merge feature only has little impact on the simulation.}}


\subsubsection{Analysis for Oncoming Lane Scenario [OLS]}
The scenario should generate many collisions and situations where planners drive on the oncoming lanes.

\paragraph{Qualitative Analysis}\mbox{}\\
During the simulation planners often drive into the oncoming lane when overtaking or changing lanes,
but this rarely results
in collisions or critical interactions with traffic participants from the oncoming lane.

\noindent\fbox{\parbox{\textwidth}{\textbf{Summary:} The OLS partly fulfills the goal as
planners drive on the oncoming lane but it rarely results in critical interactions and therefore
the oncoming lane feature only has little impact on the simulation.}}

%\section{Discussion}\label{sec:discussion}


\section{Threats to Validity}\label{sec:threats-validity}

\subsection{External Validity}\label{subsec:external-validity}
The evaluation was only carried out with four scenarios, it could be that the generation
of critical interactions cannot be generalised to other scenarios.
To mitigate this risk, the scenarios were selected so that each scenario contains different
road features(\ref{subsec:base-scenario-selection}), this way the effects of the different features
could be analysed(\ref{subsubsec:scenario-effects}), which increases the possibility of generalisation.

\subsection{Internal Validity}\label{subsec:internal-validity}
Internal validity can be affected by the randomness of evolutionary
algorithms~(\ref{subsec:evolutionary-algorithm-1+1}), which leads
to a different result for each randomly generated scenario, thereby potentially skewing the overall result.
To minimise the risk of randomness, a large number of simulations were executed, \numberRuns~runs for each
scenario--algorithm combination and
\numberSimulationsPerRun~simulations per run result in overall 24.000 simulations(~\ref{sec:experimental-setup}).

Another issue is the possibility of bugs in the implementation, which can have many effects, e.g.\ that
the MPP does not recognise a collision of two planners or that metrics are recorded incorrectly.
To counteract this issue, the code was thoroughly checked and the generated
GIFs of the simulations were sampled for validity after each run.

\subsection{Construct Validity}\label{subsec:construct-validity}
An interaction is defined as a change in trajectory compared to the previously
computed trajectory and another ego vehicle being within the safety distance(\ref{subsubsec:interaction-def}).
The issue here comes when checking for an interaction, as a change in trajectory
does not necessarily come from an interaction, as planners can also change the trajectory
for readjustment purposes without an interaction happening,
this can lead to false positive interactions.
Due to the implementation effort and the long runtimes of other solutions, this problem is tolerated,
and furthermore, the effect
is limited by the second requirement of another ego within the safety distance, so not every trajectory
adjustment is counted as an interaction.
