\chapter{Background and State of the Art}\label{ch:background}

%TODO design concepts good word? gesucht ist ein genereller begriff für evolutionäre algos
This chapter briefly describes the current state of the art for testing autonomous vehicles
and introduces the most common design concepts.
The second section gives an overview about the technologies used for the realisation of this thesis and also
the implementation of the MPP\@.


\section{Background}\label{sec:background}

\begin{figure}
    \centering
    \includegraphics[width=0.9\linewidth]{images/helper-visualizations/reactive-planner-sampling}
    \caption{Regular interval sampling compared to the set-based sampling of the Reactive
    Planner\cite{wursching2021sampling}.}
    \label{fig:diagrams:sampling-visualization}
\end{figure}

\subsection{Sampling-based Motion Planning}\label{subsec:sampling-based-motion-planning}
Basically, motion planning deals with solving the problem of moving an object from a starting
point to a destination point using a sequence of valid actions.
This makes motion planning a core component of autonomous vehicles and since
the problem to be solved is non-trivial, there are several different approaches used in planners,
e.g.\ graph search-based planners, interpolating curve planners or sampling-based planners\cite{gonzalez2015review}.

In this thesis, a Sampling-Based Motion Planner is used.
These planners generate a variety of random valid, and in some algorithms also invalid,
trajectories that approach or reach the goal.
The best trajectory is then selected from this set of possible trajectories.
The generation and selection of trajectories varies from algorithm to algorithm.
In \autoref{fig:diagrams:sampling-visualization} (a) the generated trajectories and the selected
best trajectory are shown.\\
Sampling-based planners are not always guaranteed to reach the goal
and the paths found are often suboptimal, but an advantage is that path finding is fast
compared to other MPAs\cite{karaman2011sampling,elbanhawi2014sampling}.

\subsection{CommonRoad}\label{subsec:commonroad}
The newly developed MPP is based on the \CommonRoad~framework, which is short for the descriptive name
\textit{composable benchmarks for motion planning on roads}\footnote{\url{https://commonroad.in.tum.de/} Date: \FootDate}.
The goal of \CommonRoad~is to give researchers a platform to compare their MPAs and be able to generate reproducible
results for others, furthermore this also saves time for all kind of simulation preparations\cite{althoff2017commonroad}.
Using this framework, research on testing MPAs has already been
conducted, and the studies have shown that \CommonRoad~can be used to generate a variety of
critical scenarios for testing\cite{Klischat2019a, klischat2019generating, Klischat2020c}.
All these exemplary researches are testing an MPA and trying to make it take faulty actions by generating
more and more critical scenarios.
The criticality in these studies comes from using an evolutionary algorithm\cite{klischat2019generating}
or combining the simulation with another tool to generate
realistic traffic participants\cite{Klischat2019a, Klischat2020c}.


\begin{figure}
    \centering
    \includegraphics[width=0.7\linewidth]{images/helper-visualizations/scenario-visualization}
    \caption{Scenario with several dynamic obstacles(blue) and a planning problem(green) on a
    four way intersection with the goal(yellow) to perform a turn to the right\footnotemark.}
    \label{fig:diagrams:scenario-visualization}
\end{figure}
\footnotetext{Scenario DEU\_Muc-16\_1\_I-1-1 from https://commonroad.in.tum.de/scenarios; Date: \FootDate}

\subsubsection{Scenario}\label{subsubsec:scenario}
An initial scenario consists of a road network, also called lanelets or lanelet network, a set of planning problems
and optionally dynamic or static obstacles.
With an MPA a scenario can be simulated over a configured amount of time
steps and each traffic participant progresses accordingly to the time steps.
For this thesis dynamic and static obstacles are not in use and instead multiple planning problems are being used.
Also traffic signs and other details associated with the road network are not used.
%TODO what is a single lanelet


\subsubsection{State of Ego Vehicle}\label{subsubsec:state}
A state describes position, velocity, angle, acceleration and other more detailed information of the
ego vehicle at a given time step.
The more detailed information in the state is only relevant to the planner but not to the implementation of the
MPP\@.


\subsubsection{Planning Problem}\label{subsubsec:planningproblem}
A Planning Problem consists of an initial state, which is a regular state (\ref{subsubsec:state}), and a goal area.
The goal area can be located at any point in the road network of the scenario and can have any size or
valid two-dimensional shape.
The validation of whether a goal can be reached from the initial state must be implemented separately
and is not predefined by \CommonRoad.
The scenario visualization of \autoref{fig:diagrams:scenario-visualization} shows the initial state of
a \PlanningProblem~as a green arrow and the goal area as a yellow rectangle.

Apart from that, a goal does not have to be an area, but can also have a range requirement for velocity, time
steps or lanelets.
However, this case is not used in this work because the focus
is on the interactions and these goal definitions are not beneficial for this.


\subsubsection{Trajectory}\label{subsubsec:trajectory}
A trajectory is considered as a sequence of states, each state being incrementally
one time step greater than its predecessor, gaps in the time sequence are invalid.
A trajectory can have any length and starts with the current time step of the simulation.

As the simulation advances with each time step, new trajectories must be computed repeatedly by each planner
and in the case of the MPP also shared with the other planners.

\subsubsection{Dynamic/Static Obstacles}\label{subsubsec:dynamic-obstacles}
In the \CommonRoad~domain, dynamic obstacles are cars with a trajectory predetermined for a complete scenario,
the cars are only included in the scenario during the specified trajectory,
outside the time interval they are not present at all.
In the scenario visualization of \autoref{fig:diagrams:scenario-visualization}
the blue cars represent dynamic obstacles.

Static obstacles, on the other hand, have no trajectory
and are in the same place in the scenario throughout the simulation.

\subsubsection{CommonRoad Tool Set}\label{subsubsec:tool-set}
\CommonRoad~offers different tools to work with scenarios and build software on top of the \CommonRoad~framework.
One of those tools is the \CommonRoad~Scenario Designer, which can be used to view and sketch scenarios, this
helps with easy creation of simple test case scenarios during development.


\subsubsection{Reactive Planner}\label{subsubsec:reactive-planner}
For testing, developing and later also evaluating the novel multi planning platform an MPA is needed.
Therefore, the planner from\cite{wursching2021sampling} is used.
This planner has been developed on the \CommonRoad~platform, which allows a good integration into the MPP\@.
The current state of the planner mainly focuses on highway scenarios.

This planner takes a new approach to calculating trajectories by reducing the search space of a sampling-based planner
to produce only sets that can be reached without collision.
As a result, this approach can reduce the number of samples needed and drastically reduce the computation
time of the planner to find a feasible trajectory, which is very effective especially in critical situations\cite{wursching2021sampling}.
The visualization of \autoref{fig:diagrams:sampling-visualization} (b) clearly shows the improvements in trajectories,
with less off-road and collision-free trajectories compared to an interval sampling approach.


\subsubsection{Collision Checker}\label{subsubsec:collision-checker}
The tool Collision Checker\footnote{\url{https://commonroad.in.tum.de/docs/commonroad-drivability-checker/doxygen/html/index.html} Date: \FootDate},
also provided by \CommonRoad,
is an elementary component of the Reactive Planner and the MPP, as it
checks whether geometric shapes collide and can also be applied to dynamic/static objects.
The Collision Checker works by adding all objects of a scenario to itself
and can then check whether collisions are present,
this is particularly relevant for the MPP, as collision checks are required at every time step of a simulation.
It is also heavily used by the Reactive Planner to compute the next trajectories as collision-free as possible.


\subsection{Evolutionary Algorithm}\label{subsec:evolutionary-algorithm}
To generate improved scenarios, an evolutionary algorithm is used to obtain
more critical situations in the simulations.
An evolutionary algorithm typically operates with a population of individuals that compete with
each other in certain resources, this competition then increases the fitness
of the population, i.e.\ the population improves in obtaining the resources\cite{eiben2015evolutionary}.
With fitness and random mutations on the individuals of a population, better populations can be created.

In this work, individuals are represented by \PlanningProblems, each of which has a Reactive Planner assigned to it;
these occur in a scenario and are therefore the population.
The resources in this case are the defined critical metrics and are generated by the planners in a simulation.
The resources determine how good a simulation of a scenario is.
To improve the simulations, fitness functions are defined that lead the planners to more critical situations.
Between simulations, the initial states of the planning problems are mutated.

\subsection{Evaluation of Evolutionary Algorithms}\label{subsec:evaluation-of-evolutionary-algorithms}
To compare the fitness functions of the evolutionary algorithm, the procedure from\cite{arcuri2014hitchhiker} is used.
This research serves as a guideline for the statistical evaluation of randomized algorithms and is widely accepted
amongst researchers.

For this purpose, the A12 effect mass and the corresponding p-value for each result are reported and
computed according to\cite{arcuri2014hitchhiker}.
The values of the A12 effect mass represent the difference between two populations and
can be interpreted as follows\cite{vargha2000critique}:
\begin{itemize}
    \setlength\itemsep{-0.7em}
    \item Big: A12 >= 0.71
    \item Medium: A12 >=0.64
    \item Small: A12 >= 0.56
\end{itemize}


\section{State of the Art}\label{sec:state-of-the-art}

\subsection{Testing Autonomous Vehicles}\label{subsec:testing-autonomous-vehicles}

AD is becoming increasingly complex and therefore needs to be tested more effectively.
Since autonomous vehicles cannot be tested like conventional vehicles, they must be
tested in a large number of realistic and diverse driving scenarios.
To achieve this, the tests are often carried out in realistic simulations.
In most studies, autonomous vehicles are tested individually in a simulation,
where other traffic participants or road networks are virtually generated\cite{huang2016autonomous}.
This allows a quick execution of a large number of different scenarios and saves a lot of time compared
to manual scenario selection.
In order to guarantee not only quantity but also quality of the scenarios, the generation is often
combined with an evolutionary algorithm, as in these examples\cite{klischat2019generating,gambi2019automatically,li2020av},
this saves a lot of time and the unnecessary execution of duplicate or uninteresting tests.
Therefore, newly generated scenarios are evaluated by a fitness function and modified until a critical behaviour
of the autonomous vehicle is found, which can be, for example, leaving the road as in\cite{gambi2019automatically}.
Approaches like these are designed to find a specific type of misbehaviour and are finished when a critical behaviour is found.
%TODO more info needed
%TODO add info especially about MPAs or testing MPA
