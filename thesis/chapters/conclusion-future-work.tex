\chapter{Conclusions and Future Work}\label{ch:conclusion-future-work}

\section{Conclusions}\label{sec:conclusions}

Testing AVs raises a number of various challenges, including the amount of tests and critical scenarios required
to properly test AVs under as many diverse situations as possible\cite{koopman2016challenges}.
A variety of virtual test generation methods are used to address this in other works\cite{huang2016autonomous},
but testing how multiple AVs interact with each other is almost completely neglected.
Therefore, this thesis presents the MPP, a platform that generates critical interactions for
MPAs by making multiple planners interact with one another.

The experiment shows that with the MPP and an evolutionary algorithm, a large number of critical situations
are found due to interactions.
Furthermore, the defined fitness metrics are particularly good at generating critical situations.
As a result, different critical behaviours are found in different scenarios with the planner under test.
Besides that, the selected scenarios did not show any statistically relevant differences in the number of individual
critical metrics generated, but the slightly increased values suggest to still select scenarios with certain
features to generate critical situations for a specific use case.

Although many critical situations are generated, there are also limitations in the generation, for example, the
criticality of some scenarios is based on one or more planners coming to a standstill early in the simulation
and blocking the road for upcoming ego vehicles.
This behaviour occurs in both presented algorithms and reduces the quality
of the scenarios, as scenarios with static obstacles are not very interesting
and other research with explicitly adding static obstacles has already been conducted\cite{hu2018dynamic}.

Another limit is the definition of an interaction, because as mentioned
in Definition~\nameref{subsubsec:interaction-def},
this also evaluates false positive interactions.
This affects the performance of the \MaxInteractions~algorithm and can generally distort the tracking of interactions.
However, the effect is not substantial enough to affect the result, but nevertheless a
more precise measurement is always better and more robust.
%TODO are there more limitations?

\section{Future Work}\label{sec:future-work}
The novel MPP provides a good foundation for investigating on interactions with multiple planners and gives
insights into how AVs may interact on the road of a real world scenario.
But there are a lot of other things to add to this project, which is beyond the scope of a Master's thesis:

A scenario can achieve good critical values in a simulation and still have poor quality if the critical
interactions are uninteresting (e.g.~\nameref{par:tailgating-interaction}(3)).
This can be counteracted by categorising interactions directly live in the simulation, which
can be achieved by analysing the last states of the involved ego vehicles.

Mutation algorithms were evaluated purely based on the criticality metrics and were kept very simple.
As the algorithms have shown that they are able to produce bad behaviours in the planner, they can also
be refined and improved to incorporate different metrics.
This can improve the scenario generation to find scenarios of higher critical quality instead of e.g.\ uninteresting
tailgating behaviours~\ref{par:tailgating-interaction}.
In addition, the fitness functions can be redefined to improve the quality already during the simulation
or the generation of the scenarios.
Approaches here could include a combination of the Danger and Interaction metrics or,
to improve scenario generation, planners with little impact on the simulation can be removed
or mutated with higher probability.
%
%
%Improve data collection:
%Goals not reached metric is not very meaningful as it is affected by collisions and emergency maneuvers
%for the future those two cases should be deducted from goals not reached and this metric can then for example
%show that goals where either missed due to limited amount of time steps or because the planner did not manage
%to drive in the goal area.
%
%One other interesting point would be if cars are able to recover from the emergency maneuvers and reach the goal,
%this can not be concluded from the current data.
%
%%TODO for more interesting insights and pages in evaluation
%A way to accomplish this with current data would be to divide EM by configured_max_EM_per_ego and then this can be
%compared to goals_not_reached - collisions, actually this should be possible with current data!
%Actually this seems not to be possible because the collision metric is not tracked very well.
%1 collision means 2 cars collided, but a third car could also crash into the 2 already collided cars which means there
%is now 2 collisions with 3 cars involved but 2 collisions could also be 2 separate collisions which would
%then be 4 cars involved
