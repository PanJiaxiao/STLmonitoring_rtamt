% !TeX spellcheck = en_US
% !TeX encoding = UTF-8
\chapter*{Abstract}

As motion planning is an essential part of autonomous driving and autonomous vehicles (AVs) in general,
it must be thoroughly tested in order
to meet the high safety requirements.
Testing in this domain is mostly done within simulations on predefined scenarios with specific
parts of a road network and traffic participants following a predefined trajectory, the goal for instance
may be, to test
a motion planning algorithm (MPA) in many diverse and challenging scenarios and make the planner perform safety
critical maneuvers.
One of the main problems of this approach is how to generate many
critical and difficult scenarios.\\
To overcome this issue, testing is often automated to generate many difficult scenarios.
With the focus on the generation of scenarios, one aspect is neglected and that is the interaction of AVs with
each other, which can lead to unpredictable interactions that can also end in accidents.
Since more and more AVs will be on the roads in the future and there are already first experiments
with AVs as driverless taxis, this topic is becoming increasingly important.\\
In this thesis a new approach of scenario generation is proposed: A system, based on the \CommonRoad~framework,
is developed to automatically simulate multiple instances of the same MPA in a single scenario.
The system then relies on the interactions between the multiple MPA instances to generate critical scenarios.
To drive the simulations forward, this system is combined with an evolutionary algorithm that mutates the scenario under
test in between simulations.
The experiments show that the interaction-based system, in combination with mutation algorithms, is
able to find several critical scenarios that lead to faulty interactions between planners.
